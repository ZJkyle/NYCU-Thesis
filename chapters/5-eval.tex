\chapter{Evaluation}

本章節詳細評估本研究所提出的多SLM協作架構的性能表現,包括實驗環境設定、性能比較分析以及系統參數優化結果。

\section{實驗環境設定}

本研究的實驗環境配置如表~\ref{table:experiment_env} 所示。我們使用了多種邊緣設備進行測試,包括Raspberry Pi 5、Jetson Orin Nano以及一般開發板,以驗證多SLM協作架構在不同硬體配置下的適用性。

\begin{table}[h]
\centering
\caption[實驗環境配置]{實驗環境硬體與軟體配置表}
\label{table:experiment_env}
\begin{tabular}{lll}
\toprule[1.1pt]
配置項目 & 規格 & 說明\\
\midrule[1.1pt]
\multirow{3}{*}{硬體配置} & CPU & Intel Core i7-12700K, 12核心\\
& 記憶體 & 32GB DDR4-3200\\
& 儲存設備 & 1TB NVMe SSD\\
\midrule[1.1pt]
\multirow{3}{*}{邊緣設備} & Raspberry Pi 5 & 8GB RAM, ARM Cortex-A76\\
& Jetson Orin Nano & 8GB RAM, ARM Cortex-A78AE\\
& 開發板 & 4GB RAM, ARM Cortex-A53\\
\midrule[1.1pt]
\multirow{3}{*}{軟體環境} & 作業系統 & Ubuntu 22.04 LTS\\
& Python版本 & Python 3.9.7\\
& 深度學習框架 & PyTorch 2.0.1\\
\midrule[1.1pt]
\multirow{2}{*}{SLM模型} & 模型1 & DeepSeek-R1-Distill-Llama-8B\\
& 模型2 & Phi-2-2.7B\\
\bottomrule[1.1pt]
\end{tabular}
\end{table} 

\section{性能比較分析}

表~\ref{table:performance_comparison} 展示了多SLM協作架構與其他方法的性能比較結果。從表中可以看出,本研究所提出的方法在準確率方面相比單一SLM方法有17.4\%的提升,同時保持了較低的延遲和合理的資源使用。

\begin{table}[h]
\centering
\small
\caption[性能比較結果]{不同方法在邊緣推理任務上的性能比較表}
\label{table:performance_comparison}
\begin{tabular}{lccccc}
\toprule[1.1pt]
方法 & 準確率(\%) & 延遲(ms) & 記憶體(GB) & CPU(\%) & 吞吐量\\
\midrule[1.1pt]
單一SLM & 72.3 & 850 & 6.2 & 85 & 1.2\\
\multirow{2}{*}{多SLM協作} & 89.7 & 920 & 7.8 & 92 & 1.1\\
& (+17.4\%) & (+8.2\%) & (+25.8\%) & (+8.2\%) & (-8.3\%)\\
\midrule[1.1pt]
雲端推論 & 95.2 & 2150 & 0.5 & 15 & 0.5\\
\midrule[1.1pt]
傳統分散式 & 78.5 & 1250 & 8.5 & 95 & 0.8\\
\bottomrule[1.1pt]
\end{tabular}
\end{table} 

\section{任務分解粒度分析}

不同硬體配置下的最佳任務分解粒度分析結果如表~\ref{table:task_decomposition} 所示。結果顯示,硬體配置越強大,可以採用更細粒度的任務分解策略,從而獲得更高的整體效率提升。

\begin{table}[h]
\centering
\small
\caption[任務分解粒度分析]{不同硬體配置下的最佳任務分解粒度分析表}
\label{table:task_decomposition}
\begin{tabular}{lccccc}
\toprule[1.1pt]
硬體配置 & 記憶體(GB) & 最佳粒度 & 子任務數 & 聚合時間(ms) & 效率提升(\%)\\
\midrule[1.1pt]
Raspberry Pi 5 & 8 & 中等 & 3-4 & 45 & 23.5\\
\midrule[1.1pt]
Jetson Orin Nano & 8 & 細粒度 & 5-6 & 38 & 31.2\\
\midrule[1.1pt]
開發板 & 4 & 粗粒度 & 2-3 & 52 & 18.7\\
\midrule[1.1pt]
桌面系統 & 32 & 極細粒度 & 8-10 & 25 & 42.8\\
\bottomrule[1.1pt]
\end{tabular}
\end{table} 

\section{系統參數設定}

多SLM協作架構的系統參數配置如表~\ref{table:system_parameters} 所示。這些參數經過精心調優,以在不同環境下達到最佳的性能表現。

\begin{table}[h]
\centering
\small
\caption[系統參數設定]{多SLM協作架構的系統參數配置表}
\label{table:system_parameters}
\begin{tabular}{llll}
\toprule[1.1pt]
參數類別 & 參數名稱 & 預設值 & 說明\\
\midrule[1.1pt]
\multirow{3}{*}{任務分解} & 最小粒度 & 0.1 & 任務分解的最小粒度閾值\\
& 最大粒度 & 0.8 & 任務分解的最大粒度閾值\\
& 自適應係數 & 0.5 & 自適應調整係數\\
\midrule[1.1pt]
\multirow{3}{*}{協作聚合} & 聚合超時 & 5000ms & 聚合操作的最大等待時間\\
& 重試次數 & 3 & 失敗任務的重試次數\\
& 一致性閾值 & 0.8 & 結果一致性檢查閾值\\
\midrule[1.1pt]
\multirow{3}{*}{資源管理} & 記憶體閾值 & 85\% & 記憶體使用率警告閾值\\
& CPU閾值 & 90\% & CPU使用率警告閾值\\
& 負載均衡 & 輪詢 & 負載均衡策略\\
\midrule[1.1pt]
\multirow{2}{*}{性能優化} & 快取大小 & 100MB & 中間結果快取大小\\
& 並行度 & 4 & 最大並行處理數量\\
\bottomrule[1.1pt]
\end{tabular}
\end{table} 

\section{實驗結果分析}

實驗結果顯示,本研究所提出的多SLM協作架構在邊緣推理任務上表現優異。相比單一SLM方法,該架構能夠顯著提升推理準確率,同時保持較低的延遲和合理的資源消耗。特別是在複雜推理任務上,多SLM協作架構的優勢更加明顯。

此外,自適應粒度任務分解機制能夠根據不同的硬體配置動態調整策略,確保在各種邊緣設備上都能獲得良好的性能表現。協作式聚合機制則確保了結果的穩健性和一致性,即使在部分節點出現問題的情況下仍能提供可靠的服務。 