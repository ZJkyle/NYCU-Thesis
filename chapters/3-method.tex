\chapter{Proposed Method}

本章節詳細介紹本研究所提出的多SLM協作架構的設計與實現方法。該架構旨在解決邊緣設備上SLM推理準確率不足的問題,通過自適應任務分解和協作聚合機制來提升整體性能。

\section{系統架構概述}

本研究所提出的多SLM協作架構採用主從式設計,如圖~\ref{fig:system_architecture} 所示。該架構包含一個主節點和多個工作節點,通過智能任務分解和協作聚合機制來實現高效的邊緣推理。

\begin{figure}[h]
  \centering
  \includegraphics[height=!,width=0.8\linewidth,keepaspectratio=true]%
  {figures/System_arch.png}
  \caption[多SLM協作架構系統設計]{多SLM協作架構系統設計圖,展示主節點與工作節點之間的協作關係以及任務分解與聚合流程}
  \label{fig:system_architecture}
\end{figure}

系統架構的主要組件包括:

\begin{itemize}
    \item \textbf{主節點(Master Node)}:負責接收用戶查詢、進行任務分解、協調工作節點執行,以及聚合最終結果
    \item \textbf{工作節點(Worker Nodes)}:執行分配的子任務,運行不同的SLM模型,並將中間結果返回給主節點
    \item \textbf{任務分解器(Task Decomposer)}:根據硬體配置和任務複雜度,將用戶查詢分解為適合的子任務
    \item \textbf{結果聚合器(Result Aggregator)}:整合來自多個工作節點的中間結果,生成最終輸出
\end{itemize}

\section{自適應粒度任務分解機制}

自適應粒度任務分解是本架構的核心創新之一。該機制能夠根據邊緣設備的硬體配置(如記憶體大小、CPU核心數)動態調整任務分解的粒度,以達到最佳的推理效率。

任務分解過程包括以下步驟:

\begin{enumerate}
    \item \textbf{硬體資源評估}:分析當前設備的可用記憶體、CPU使用率等資源狀況
    \item \textbf{任務複雜度分析}:評估用戶查詢的複雜程度,包括文本長度、推理深度等
    \item \textbf{粒度決策}:根據資源狀況和任務複雜度,決定最適合的分解粒度
    \item \textbf{子任務生成}:將原始查詢分解為多個可以並行執行的子任務
\end{enumerate}

\section{協作式聚合機制}

協作式聚合機制確保了多個SLM模型協作的有效性。該機制採用主從式架構,主節點負責協調和聚合,工作節點專注於執行特定的推理任務。

聚合過程的主要特點:

\begin{itemize}
    \item \textbf{容錯性}:即使部分工作節點出現問題,系統仍能基於可用的結果生成輸出
    \item \textbf{一致性}:通過一致性檢查機制,確保最終結果的準確性和可靠性
    \item \textbf{可擴展性}:支持動態添加或移除工作節點,適應不同的硬體配置
\end{itemize}

\section{端到端推論流程}

完整的端到端推論流程包括以下階段:

\subsection{初始化階段}
系統啟動時,主節點會掃描可用的工作節點,評估其硬體配置和SLM模型能力,建立節點註冊表。

\subsection{查詢處理階段}
當接收到用戶查詢時,主節點首先進行任務分析,然後根據自適應粒度策略將查詢分解為子任務。

\subsection{並行執行階段}
分解後的子任務被分配到不同的工作節點,各節點並行執行SLM推理,並將中間結果返回給主節點。

\subsection{結果聚合階段}
主節點收集所有工作節點的中間結果,通過協作聚合機制生成最終輸出,並返回給用戶。

\section{實現細節}

本架構基於Llama.cpp框架實現,充分利用了其高效的推理引擎和靈活的模型載入機制。主要實現特點包括:

\begin{itemize}
    \item \textbf{輕量級通信}:使用高效的RPC協議進行節點間通信,最小化網絡開銷
    \item \textbf{動態負載均衡}:根據各節點的當前負載情況,動態調整任務分配
    \item \textbf{資源監控}:實時監控各節點的資源使用情況,確保系統穩定性
\end{itemize} 