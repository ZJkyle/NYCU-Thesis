\chapter{thesis.cls 簡介}
\label{chapter:doccls}

\textbf{thesis.cls} 是本 template 的核心, 其主要功能是產生論文的封面, 設定版面與章節樣式, 並提供自動選擇字型的功能.
因大部分需要使用者輸入的資訊(e.g. author name)都有拉出對應的變數讓使用者設定, 因此
絕大部分的情況下, 你是不需要動到 thesis.cls 的.
當然, 如果你有發現 bug 或是覺得哪些地方有更好的寫法, 也請麻煩在 github 上發個 issue 告訴我吧, 謝謝!

以下介紹如何設定封面資訊與字型, thesis.cls 提供的 options, 以及摘要等檔案放置的位置.

\section{封面資訊}

封面資訊的設定指令全部都放在 \textit{config/frontmatters.tex} 中, 請在此設定你論文的 title, author name, (co-)advisor name, college name (院), institute name (所), field (領域), 日期等資訊.

\textbf{注意:} 如果沒有共同指導教授, \textbackslash coadvisorA, \textbackslash coadvisorAzh, \textbackslash coadvisorB \textbackslash coadvisorBzh 的\{ \}請留白.

\section{字型設定}
\label{sec:fonts}

為了避免在不同作業系統上編譯文件時要重新設定中文字型, thesis.cls 提供了自動選擇字型的功能.
(請在 xelatex 加入 \textbf{-shell-escape} 才能開啟此功能)

\subsection{預設英文字型}

thesis.cls 預設的英文字型在全部作業系統上皆為
\begin{itemize}
\item 本文字型(main font): \textbf{Times New Roman}
\item 無襯線字型(sans-serif font): \textbf{Arial}
\end{itemize}

\subsection{預設中文字型}

thesis.cls 預設的中文字型則詳列於 Table~\ref{table:zhfonts}.
簡單來說, 本文使用``標楷體", 封面字型則使用``明體".

% table 
\begin{table}[h]
\centering
% [] 顯示在 list of tables 的文字
% {} 顯示在表格上方的文字
\caption[Default Chinese font settings]{預設的中文字型}
\label{table:zhfonts}
\begin{tabular}{@{}lll@{}}
\toprule
作業系統     & 本文字型     & 明體字 (用於封面) \\ \midrule
Windows  & 標楷體      & 新細明體       \\
Linux    & AR PL 中楷 & AR PL 明體   \\
Mac OS X & 楷體-繁     & 儷宋 Pro     \\ \bottomrule
\end{tabular}
\end{table}

\subsection{修改預設字型}

thesis.cls 提供了以下四個修改預設字型的指令.
\textbf{注意: 如果修改了預設字型, thesis.cls 則不會再根據所處的作業系統自動選擇字型.}
我將此四個指令獨立出來放在 \textit{config/fonts.tex} 裡面.
如果不想修改預設字型, \{ \} 請留白.

\begin{itemize}

\item \textbf{\textbackslash mainfontzh\{\}} 修改預設中文本文字型

\item \textbf{\textbackslash mingfontzh\{\}} 修改預設中文明體字型

\item \textbf{\textbackslash mainfont\{\}} 修改預設英文本文字型

\item \textbf{\textbackslash sansfont\{\}} 修改預設英文無襯線字型

\end{itemize}


\section{Options provided by thesis.cls}

thesis.cls 提供碩士與博士論文模板, 並提供初稿與終稿等選項讓使用者自行印出的版面格式.
thesis.cls 的 options 設定就在 \textbf{main.tex} 的第一行

\textbackslash documentclass[\textbf{<options>}]\{thesis\}

以下介紹 thesis.cls 所提供的 options

\newpage

\subsection{給沒空的人看的無腦版本}

\paragraph{英文碩士論文} 請設定

\begin{itemize}
\item 初稿\\
\textbackslash documentclass[]\{thesis\}
\item 終稿\\
\textbackslash documentclass[watermark,final]\{thesis\}
\item 只顯示論文內文, 附錄, 和 reference\\
\textbackslash documentclass[review]\{thesis\}
\end{itemize}

\paragraph{中文碩士論文} 在 [ ] 中多加一個 `zh' 就會顯示中文的章節編號和標題了, 舉例來說

\begin{itemize}
\item \textbackslash documentclass[zh,watermark,final]\{thesis\}
\end{itemize}


\paragraph{英文博士論文} 請在 [ ] 中多加一個 `phd', 舉例來說

\begin{itemize}
\item \textbackslash documentclass[phd,watermark,final]\{thesis\}
\end{itemize}

\paragraph{中文博士論文} 一樣在 [ ] 中多加一個 `zh' 就可以了, 舉例來說

\begin{itemize}
\item \textbackslash documentclass[phd,zh,watermark,final]\{thesis\}
\end{itemize}

\paragraph{雙面列印} 預設為單面列印, 如果要雙面列印, 請在 [ ] 中加入 'twoside', 比方說

\begin{itemize}
\item \textbackslash documentclass[twoside,phd,watermark,final]\{thesis\}
\end{itemize}

\paragraph{Known Issue!} 在編譯過英文版後, 加入 `zh' 編譯中文版後的\textbf{第一次編譯}時, XeLaTeX 會出錯.
但只要再編譯一次就沒問題了.
也就是說從英文版轉到中文版的編譯步驟變成:

\hspace{2em} \textbf{XeLaTeX} (\textbf{\textit{Failed!}}) $\rightarrow$ \textbf{XeLaTeX} $\rightarrow$ \textbf{BibTeX} $\rightarrow$ \textbf{XeLaTeX} $\rightarrow$ \textbf{XeLaTeX}

\subsection{詳細版的 Class Options}

thesis.cls 提供的 options 詳列於 Table~\ref{table:clsoptions}.
標注 (default) 代表 thesis.cls 的預設值, 不需要寫在 [ ] 內.

% 把 table 的內容放在另一個檔案再 load, 讓 tex 看起來乾淨一點
\input{tables/table-classopt}

\section{論文檔案位置}

Table~\ref{table:pages} 詳列了摘要, 誌謝等頁面的位置.
如果你使用 Texmaker 之類的 editor 的話, 可以從 main.tex 連結到這些檔案.

\textbf{注意:} 題獻頁, 自傳, 著作目錄只有博士論文才需要填寫, 也只有在 \textbackslash documentclass[] 加入 `phd' 選項後才會被編譯.

\input{tables/table-files}

至於論文內文的檔案要放哪裡, 則看個人喜好.
我自己的習慣是把論文的每一個 chapter 獨立成一個 .tex 檔, 放在 \textit{chapters/} 下.
而附錄也是寫成獨立的 .tex 檔, 放在另一個資料夾下, 方便管理.
請參考此 template 的 \textit{chapters/} 與 \textit{appx/} 兩個資料夾.

\chapter{背景與相關工作}

\section{背景}
\subsection{大型語言模型的挑戰}
大型語言模型(LLMs)的突破推動了多領域的應用,例如GPT、Gemini等,但龐大的推理成本成為限制應用普及的主要瓶頸。現有解決方案可分為訓練階段的參數擴展與推理階段的計算資源放大,前者成本極高,後者則仍有潛力可挖掘,例如lensemble、推理協作、分布式推理等方法。

\subsection{On-device與Edge端AI的需求與現況}
隨著小型LLM性能大幅提升(如Deepseek-R1-Distilled-Qwen-1.5B等),Edge/本地設備上執行複雜任務成為可能,但在數據密集型推理、長文本推理等挑戰下,單靠本地模型仍難以達到雲端前沿模型的品質。現實場景需求包括成本敏感、隱私要求、延遲限制等,使得「邊緣-雲端協作」成為重要趨勢。

\subsection{MoA推論}
多層代理(MoA, Multi-Layer Agents)推論方法近年受到關注,包含基本MoA、Self-MoA等架構。這些方法在回應選擇、早停、角色分配等方面各有優勢。下表比較了不同MoA相關方法:

\begin{table}[h]
\centering
\caption{MoA方法比較}
\begin{tabular}{|l|l|l|}
\hline
Paper & Concept & Link \\
\hline
MoA & Multi-Layer Agents inference & \url{https://arxiv.org/abs/2406.04692} \\
SMoA & Response Selection, Early Stopping, Role Assignment & \url{https://arxiv.org/abs/2411.03284} \\
Self-MoA & agents re-sampling & \url{https://arxiv.org/abs/2502.00674} \\
\hline
\end{tabular}
\end{table}

\section{相關工作}
\subsection{強化學習與監督式微調}
現有如Deepseek R1、Open R1等方法,結合強化學習與監督式微調以提升模型表現。

\subsection{推論最佳化}
推論最佳化技術如剪枝(pruning)與量化(quantization),能有效降低模型運算需求,提升推論效率。

\subsection{叢集推論}
叢集推論(Cluster inference)則透過多設備協作,進一步提升推論效能與可靠性。
