\chapter{Figures and Tables}
\label{chapter:fig}

本章節展示本研究所提出的多SLM協作架構相關的圖表和表格,包括系統架構圖、實驗結果圖以及性能比較表格。

\section{系統架構圖}

圖~\ref{fig:system_architecture} 展示了本研究所提出的多SLM協作架構的整體設計。該架構採用主從式設計,包含一個主節點和多個工作節點,通過自適應任務分解和協作聚合機制來提升邊緣推理的準確率。

% 系統架構圖已在第三章中展示,這裡不再重複

\section{任務分解流程圖}

% 暫時註釋掉,等圖片準備好後再取消註釋
% 圖~\ref{fig:task_decomposition} 詳細說明了自適應粒度任務分解的過程。該機制根據邊緣設備的硬體配置動態調整任務分解的粒度,以達到最佳的推論效率。

% 暫時註釋掉,等圖片準備好後再取消註釋
% \begin{figure}[h]
%   \centering
%   \includegraphics[height=!,width=0.7\linewidth,keepaspectratio=true]%
%   {figures/task_decomposition}
%   \caption[自適應任務分解流程]{自適應任務分解流程圖,展示從用戶查詢到子任務生成的完整過程}
%   \label{fig:task_decomposition}
% \end{figure}

\section{協作聚合機制圖}

% 暫時註釋掉,等圖片準備好後再取消註釋
% 圖~\ref{fig:collaborative_aggregation} 展示了協作式聚合機制的工作流程。主節點負責接收和聚合來自多個工作節點的中間結果,確保最終輸出的穩健性和一致性。

% 暫時註釋掉,等圖片準備好後再取消註釋
% \begin{figure}[h]
%   \centering
%   \includegraphics[height=!,width=0.75\linewidth,keepaspectratio=true]%
%   {figures/collaborative_aggregation}
%   \caption[協作聚合機制工作流程]{協作聚合機制工作流程圖,展示主節點如何聚合多個工作節點的結果}
%   \label{fig:collaborative_aggregation}
% \end{figure}

\section{實驗結果圖}

% 暫時註釋掉,等圖片準備好後再取消註釋
% 圖~\ref{fig:accuracy_comparison} 比較了不同方法在邊緣推理任務上的準確率表現。結果顯示本研究所提出的多SLM協作架構相比單一SLM方法有顯著的性能提升。

% 暫時註釋掉,等圖片準備好後再取消註釋
% \begin{figure}[h]
%   \centering
%   \includegraphics[height=!,width=0.8\linewidth,keepaspectratio=true]%
%   {figures/accuracy_comparison}
%   \caption[不同方法準確率比較]{不同方法在邊緣推理任務上的準確率比較圖,展示多SLM協作架構的優勢}
%   \label{fig:accuracy_comparison}
% \end{figure}

圖~\ref{fig:latency_analysis} 分析了不同方法在延遲方面的表現。多SLM協作架構在保持較高準確率的同時,能夠維持較低的推理延遲。

% 暫時註釋掉,等圖片準備好後再取消註釋
% \begin{figure}[h]
%   \centering
%   \includegraphics[height=!,width=0.8\linewidth,keepaspectratio=true]%
%   {figures/latency_analysis}
%   \caption[延遲分析結果]{不同方法在推理延遲方面的比較分析圖}
%   \label{fig:latency_analysis}
% \end{figure}

\section{資源使用分析圖}

% 暫時註釋掉,等圖片準備好後再取消註釋
% 圖~\ref{fig:resource_usage} 展示了多SLM協作架構在記憶體和CPU使用方面的效率。該架構能夠有效利用邊緣設備的有限資源,實現高效的本地推理。

% 暫時註釋掉,等圖片準備好後再取消註釋
% \begin{figure}[h]
%   \centering
%   \includegraphics[height=!,width=0.8\linewidth,keepaspectratio=true]%
%   {figures/resource_usage}
%   \caption[資源使用分析]{多SLM協作架構的記憶體和CPU使用情況分析圖}
%   \label{fig:resource_usage}
% \end{figure}
